	\documentclass{article}
	\usepackage{graphicx}
	\usepackage{amssymb}
	\usepackage{amsmath}
	\usepackage{mathrsfs}
	\usepackage{amsthm}
	\usepackage{hyperref}
	
	
	\title{Tarea II, Álgebra Lineal I\\[1ex] Dimensión y transformaciones lineales}
	\author{
		Santillán González Abraham Candelario
		\and
		Trejo Rodríguez Dara Tiferet
	}
	
	\date{Universidad Nacional Autónoma de México, Facultad de Ciencias\\[1ex] Septiembre 2025}
	
	\begin{document}
		
		\maketitle
		\newpage
		
		\section*{Nota Importante de la version}
		
		Esta versión contiene el avance parcial del documento en LaTeX.
		Por razones de las contingencias actuales, actividades correspondientes
		y el ajuste a los horarios en clases en línea,
		no fue posible concluir el formato completo antes del cierre en Classroom.
		El documento se encuentra actualmente en desarrollo. \\ 
		
		El progreso actual puede consultarse en los siguientes enlaces:
		
		\begin{itemize}
			\item GitHub: \href{https://github.com/Cande-Tlanextli/Tarea-lineal-LaTeX.git}{Repositorio GitHub}
			\item Overleaf: \href{https://www.overleaf.com/read/xvcfwkqfjvgx#c1b3df}{Proyecto Overleaf}
			\item Google Drive: \href{https://drive.google.com/drive/folders/1riBnh370KrlFX1c7xBEoZJi1Zb3alMQC?usp=sharing}{Carpeta Drive}
		\end{itemize}
		
		
		
		
		\section*{1. Dimensión}
		
		\begin{enumerate}
			
			\item \(1.1_1\) 
			Supongamos que \(U\) y \(W\) son subespacios de \(\mathbb{R}^8\) tales que \(\dim U=3\), \(\dim W=5\) y \(U+W=\mathbb{R}^8\). Prueba que:
			
			\begin{equation*}
				\mathbb{R}^8 = U \oplus W
			\end{equation*}
			
			
			\item \(1.2_2\) 
			Supongamos que \(U\) y \(W\) son subespacios de \(\mathbb{R}^9\) tales que \(\dim U=5\), \(\dim W=5\). Prueba que:
			
			\begin{equation*}
				U \cap W \neq \{0\}
			\end{equation*}
			
			\textbf{Demostración:}\\
			
			Sea \( U, W \subseteq \mathbb{R}^9\) con \(\dim U = \dim W = 5\).\\
			
			Sabemos:
			\begin{equation*}
				\dim(U+W) = \dim U + \dim W - \dim(U \cap W)
			\end{equation*}
			
			Sustituimos:
			\begin{equation*}
				\begin{aligned}
					\dim(U+W) &= 5 + 5 - \dim(U \cap W) \\
					\implies\ & \dim(U+W) = 10 - \dim(U \cap W) \\
					\implies\ & \dim(U+W) \leq 9 \\
					\implies\ & 10 - \dim(U \cap W) \leq 9
				\end{aligned}
			\end{equation*}
			
			Despejando:
			\begin{equation*}
				\begin{aligned}
					-\dim(U \cap W) \le -1 \\
					\implies\  \dim(U \cap W) \ge 1
				\end{aligned}
			\end{equation*}
			
			\begin{equation*}
				\boxed{\therefore U \cap W \neq \{0\}}
			\end{equation*}
			
		\end{enumerate}
		
		\section*{2. Transformaciones lineales}
		
		\begin{enumerate}
			
			\item[] \(2.1_5\) 
			Determina cuáles de las siguientes aplicaciones \(F\) son lineales:
			
			\item \(2.1_a\) 
			\(F: \mathbb{R}^3 \to \mathbb{R}^2\) definida por \(F(x,y,z)=(x,z)\)
			
			\item \(2.1_c\) 
			\(F: \mathbb{R}^3 \to \mathbb{R}^3\) definida por \(F(x,y,z)=(x,y,z)+(0,-1,0)\)\\
			
			Calculando:
			\begin{equation*}
				\begin{aligned}
					&F(0,0,0)  (\text{vector }0)\\
					\implies & F(0,0,0) =(0,0,0)+(0,-1,0)\\
					&\text{} = (0,-1,0)\\	
					&\text{} \neq (0,0,0)
				\end{aligned}
			\end{equation*}
			
			Aditividad:\\
			\\Tomemos dos vectores \(u=(u_1,u_2,u_3)\) y \(v=(v_1,v_2,v_3)\)
			\begin{equation*}
				\begin{aligned}
					\implies F(u+v) &=(u+v)+a\\
					&=u+v+a\\
					\implies F(u)+F(v) &=(u+a)+(v+a)\\
					&=u+v+2a
				\end{aligned}
			\end{equation*}
			
			Sea \(\lambda \in \mathbb{R}\) y \(v\) cualquiera: 
			\begin{equation*}
				\begin{aligned}
					F(\lambda v) &= \lambda v+a\\
					\lambda F(v) &= \lambda (v+a)
				\end{aligned}
			\end{equation*}
			
			Ej: tenemos \(v=(0,1,0)\) y \(\lambda =2\)
			\begin{equation*}
				\begin{aligned}
					\implies F(v) &=(0,1,0)+(0,-1,0)=(0,0,0)\\
					\implies F(2v) &=F(0,2,0)\\ 
									&=(0,2,0)+(0,-1,0)=(0,1,0)
				\end{aligned}
			\end{equation*}
			
			Si fuera lineal deberiamos obtener \((0,0,0)\), sin embargo se obtiene \((0,1,0)\), lo cual es una contradicción
			\begin{equation*}
				\boxed{ \therefore F(x,y,z)=(x,y,z)+(0,-1,0)} \text{ no es lineal.}
			\end{equation*}
			
			\item \(2.1_e\) 
			\(F: \mathbb{R}^2 \to \mathbb{R}^2\) definida por \(F(x,y)=(2x,y-x)\)\\
			
			Aditividad:\\
			\\Sea \(u=(x_1,y_1)\) y \(v=(x_2,y_2)\)
			\begin{equation*}
				\begin{aligned}
					\implies F(u+v) &=F(x_1+x_2,y_1+y_2)\\
									&=(2(x_1+x_2),(y_1+y_2)-(x_1+x_2))
				\end{aligned}
			\end{equation*}
			
			Homogeneidad:\\
			\\Sea \(\lambda \in \mathbb{R}\) y \(u=(x,y)\)
			\begin{equation*}
				\begin{aligned}
					\implies F(\lambda u) &=(2(\lambda x),\lambda_y-\lambda_x)\\
										  &=(\lambda 2x, \lambda (y-x))\\
										  &=\lambda (2x,y-x)\\
										  &=\lambda F(u)
				\end{aligned}
			\end{equation*}
			
			Cumple con aditividad y homogeneidad
			\begin{equation*}
				\boxed{\therefore F(x,y)=(2x,y-x)} \text{ es lineal.}
			\end{equation*}
			
		\end{enumerate}
		
		\begin{enumerate}
			
			\item[] \(2.2_6\) Determina cuáles de las siguientes aplicaciones son lineales:
			
			\item \(2.2_a\) 
			\(F:M_{2\times2}(\mathbb{R})\to \mathbb{R}\) definida por:
			\begin{equation*}
				F(A) = tr(A) \text{ (la traza de A)}
			\end{equation*}
			
			\item \(2.2_c\) 
			\(F:\mathbb{C}^2 \to \mathbb{C}^2\) definida por:
			\begin{equation*}
				F(z_1, z_2) = (\overline{z_1}, z_2)
			\end{equation*}
			donde \(\overline{z_1}\) es el conjugado complejo de \(z_1\)
			
			\item \(2.2_d\) 
			\(F:\mathbb{R}^3 \to M_{2\times2}(\mathbb{R})\) definida por:
			\begin{equation*}
				F(x,y,z) = 
				\begin{pmatrix}
					x & y \\
					0 & z
				\end{pmatrix}
			\end{equation*}
			
			Aditividad:\\
			Si \(u=(x_1,y_1,z_1)\) y \(v=(x_2,y_2,z_2)\)
			\begin{equation*}
				\begin{aligned}
					F(u+v) &=F(x_1+x_2,y_1+y_2,z_1+z_2)\\ 
						   &=\begin{pmatrix}
						   		x_1+x_2 & y_1+y_2 \\
						   		0       & z_1+z_2
						     \end{pmatrix}\\ 
						   &=\begin{pmatrix}
						   		x_1 & y_1 \\
						   		0   & z_1
						     \end{pmatrix} +
						     \begin{pmatrix}
						     	x_2 & y_2\\
						     	0   & z_2
						     \end{pmatrix}\\
						   &=F(u)+F(v)
				\end{aligned}
			\end{equation*}
			
			Homogenidad:\\
			Para \(\alpha \in \mathbb{R}\) y \(u=(x,y,z)\)
			\begin{equation*}
				\begin{aligned}
					\implies F(\alpha u) &=F(\alpha x, \alpha y, \alpha z)\\
										 &=\begin{pmatrix}
										 	\alpha x & \alpha y\\
										 	0        & \alpha z
										   \end{pmatrix}\\
										 &=\alpha \begin{pmatrix}
										 	x & y \\
										 	0 & z
										 \end{pmatrix}\\
										 &=\alpha F(u)
				\end{aligned}
			\end{equation*}
			
		\end{enumerate}
		
		\begin{enumerate}
			
			\item[] \(2.3_7\) Sea \(T\) una transformación lineal de \(\mathbb{R}^2\) en \(\mathbb{R}^2\) tal que \(T(1,0)=(2,1)\) y \(T(0,1)=(-1,2)\). Sea \(\mathscr{C}\) el cuadrado cuyos vértices son \((0,0),(1,0),(1,1)\) y \((0,1)\). Muestra que la imagen de este cuadrado bajo \(T\) es un paralelogramo.
			
		\end{enumerate}
		
		\begin{enumerate}
			
			\item[] \(2.4_8\) Sea \(L:\mathbb{R}^2 \to \mathbb{R}^2\) una aplicación lineal, con el siguiente efecto sobre los vectores indicados:
			
			\item \(2.4_a\) \(L(3,1)=(1,2)\) y \(L(-1,0)=(1,1)\)
			
			Suponiendo a \(L\) como : 
			
			\(
			A=
			\begin{pmatrix}
					a & b \\
					c & d
			\end{pmatrix}
			\)
			
			\begin{equation*}
				\implies L(x,y)=
				\begin{pmatrix}
					ax + by\\
					cx + dy
				\end{pmatrix}
			\end{equation*}
			
			Sabemos
			\begin{equation*}
				\begin{aligned}
					L(3,1)=(1,2) &\implies 3a+b=1 \\
						          &\implies 3c+d=2 \\
					L(-1,0) =(1,1) &\implies -a=1 \\
									&\implies a=-1 \\
									&\implies -c=1 \\
									&\implies c=-1 
				\end{aligned}
			\end{equation*}
			
			Sustituimos \(a=-1\) y \(c=-1\)
			
			\begin{equation*}
				\begin{aligned}
					\implies 3(-1)+b=1 &\implies b=4 \\
					\implies 3(-1)+d=2 &\implies d=5 \\
				\end{aligned}
			\end{equation*}
			
			Obtenemos
			\(
			A=
			\begin{pmatrix}
				-1 & 4 \\
				-1 & 5
			\end{pmatrix}
			\)
			
			\begin{equation*}
				\implies L(1,0)=A
				\begin{pmatrix}
					1 \\
					0
				\end{pmatrix}
				=
				\begin{pmatrix}
					-1 \\
					1
				\end{pmatrix}
			\end{equation*}
			
			\begin{equation*}
				\boxed{\therefore L(1,0)=(-1,-1)}
			\end{equation*}
			
			\item \(2.4_b\) \(L(4,1)=(1,1)\) y \(L(1,1)=(2,-2)\)
			
		\end{enumerate}
		
		\section*{3. Núcleo e Imagen}
		
		\begin{enumerate}
			
			\item[] \(3.1_{10}\) Para cada una de las funciones dadas en los siguientes incisos demuestra que \(T\) es una transformación lineal y encuentra las bases para \(N(T)\) e \(Im(T)\). Luego calcula la nulidad y el rango de \(T\) y verifica que la nulidad\((T)+\text{rango}(T)=\dim V\). Finalmente determina si \(T\) es inyectiva o suprayectiva.
			
			\begin{equation*}
				\text{a) } T:\mathbb{R}^3\to\mathbb{R}^2; \quad T(a_1,a_2,a_3) = (a_1+2a_2 - a_1)
			\end{equation*}
			
			Sea \(u=(u_1,u_2,u_3)\) y \(v=v_1,v_2,v_3\) y \(c \in \mathbb{R}\)
			
			Aditividad:\\
			\begin{equation*}
				\begin{aligned}
					\text{Tenemos } T(u+v) &=T(u_1+v_1,u_2+v_2,u_3+v_3) \\
										   &=((u_1+v_2) +2(u_2+v_2), 2(u_3+v_3) -(u_1+v_1)) \\
										   &=(u_1+2u_2,2u_3-u_1)+(v_1+2v_2,2v_3-v_1) \\
										   &=T(u)+T(v)
				\end{aligned}
			\end{equation*}
			
			Homogeneidad: \\
			\begin{equation*}
				\begin{aligned}
					\text{Tenemos } T(cu) &=T(cu_1,cu_2,cu_3) \\
										  &=(cu_1+2cu_2,2cu_3-cu_1) \\
										  &=c(u_1+2u_2,3u_3-u_1) \\
										  &=cT(u)
				\end{aligned}
			\end{equation*}
			
			\begin{equation*}
				\boxed{\therefore T \text{, es lineal}}
			\end{equation*}
			
			\textbf{Núcleo \(N(T):\)} \\
			
			Buscamos \(a_1,a_2,a_3\) tales que \(T(a_1.a_2,a_3=0,0) \) y obtenemos:
			
			\begin{equation*}
				\begin{cases}
					a_1+2a_2 &=0 \\
					-a_1+2a_3 &=0
				\end{cases}
			\end{equation*}
			
			\begin{equation*}
				\begin{aligned}
					\implies& -(-2a_2)+2a_3 =0 \\
					\implies&2a_2+2a_3     =0 \\
					\implies& a_2+a_3       =0 \\
					\implies& a_3           =-a_2\\
					\implies& (a_1,a_2,a_3) =(-2a_2,a_2+2a_3) 
										   =a_2(-2,1,-1) \\
					\implies& N(T)=(-2,1,-1) \text{, Nulidad=1}
				\end{aligned}
			\end{equation*}
			
			\textbf{\(Im(T)\) y rango:}
			
			Sabemos que las imagenes de la base canonica: \\
			\begin{equation*}
				\begin{aligned}
					T(e_1) &=T(1,0,0) 
						    =(1,-1) \\
					T(e_2) &=T(0,1,0) 
						    =(2,0) \\
					T(e_3) &=T(0,0,1) 
						    =(0,2)
				\end{aligned}
			\end{equation*}
			
			Entonces: \((0,2) = -2(1,-1) + 1(2,0)\). Por lo que \((1,-1),(2,0)\) son una base de \(Im(T)\) y dim \(Im(T)=2\). (rango 2)\\
			Como el codominio es \(\mathbb{R}^2\) y dim \(Im(T)=2\), se sigue que \(Im(T)=\mathbb{R}^2\)
			
			\[
			\boxed{\therefore T \text{, es sobreyectiva.}} 
			\]
			
			\begin{equation*}
				\begin{aligned}
					\implies \dim N(T) + \dim Im (T) &=1+2 \\
													 &= 3 \\
												 	 &=\dim \mathbb{R}^3
				\end{aligned}
			\end{equation*}
			
			\begin{equation*}
				\boxed{\text{\(T\), no es inyectiva, ya que \(Ker(T)\neq 0\).}}
			\end{equation*}
			
			\begin{equation*}
				\boxed{\text{\(T\), es sobreyectiva, ya que \(Im (T)=\mathbb{R}^2\).}}
			\end{equation*}
			
			\begin{equation*}
				\text{b) } T:M_{2\times 3}(\mathbb{C})\to M_{2\times 2}(\mathbb{C}); \quad 
				T\begin{pmatrix}
					a_{11} & a_{12} & a_{13}  \\
					a_{21} & a_{22} & a_{23}
				\end{pmatrix} =
				\begin{pmatrix}
					2a_{11} - 2a_{12} & a_{13} - 2a_{12} \\
					0 				  & 0
				\end{pmatrix}
			\end{equation*}
			
			\begin{equation*}
				\text{c) } T:P_n(\mathbb{R}) \to P_{n-1}(\mathbb{R}); \quad T(p) = p' + p''
			\end{equation*}
			
			Si \(p,q \in P_n\) y \(\alpha ,\beta \in \mathbb{R}\)
			
			\begin{equation*}
				\begin{aligned}
					\implies T(\alpha p + \beta q) &= (\alpha p + \beta q)' + 2(\alpha p + \beta q)'' \\
												   &= \alpha(p'+2p'') +       \beta(a' + 21'')\\
												   &=\alpha T(p) + \beta T(q)
				\end{aligned}
			\end{equation*}
			
			\textbf{Núcleo:} \\
			Sea \(p(x)=a_0+a_1x+a_2x^2+\dots +_nx^{n}\)
			\begin{equation*}
				\begin{aligned}
					\implies &p'(x)  &= a_1+2a_2x+3a_3x^2+ \dots +na_{n}x^{n-1} \\
							 &p''(x) &= 2a_2+6a_3x+12a_4x^2+ \dots +n(n-1)a-{n}x^{n-2} \\
				\end{aligned}
			\end{equation*}
			
			\begin{equation*}
				\begin{aligned}
					\implies T(p) &= q' + 2p'' \\
								  &= (a_1 + 4a_2) + (2a_2+12a_3)x + (3a_3+24a_4)x^2 + \dots + n a_n x^{n-1}
				\end{aligned}
			\end{equation*}
			
			Para que \(T(p)'=0\), todos los coeficientes deben anularse:
			
			\begin{equation*}
				\begin{aligned}
					na_n &= 0
					\implies an &=0
				\end{aligned}
			\end{equation*}
			
			Con esto el término anterior de \(a_{n-1}=0\), luego \(a_{n-2=0}\), y así sucesivamente hasta \(a_1=0\)
			
			\[
			\boxed{\therefore N(T)= \text{span}\{1\}, \dim N(T)=1.}
			\]
			
			\textbf{Imagen: }\\
			Tenemos
			\begin{equation*}
				\begin{aligned}
					p(x) &= a_0+a_1x+a_2x^2+\dots +_nx^{n} \\
					p'(x)  &= a_1+2a_2x+3a_3x^2+ \dots +na_{n}x^{n-1} \\
					p''(x) &= 2a_2+6a_3x+12a_4x^2+ \dots +n(n-1)a-{n}x^{n-2} \\
				\end{aligned}
			\end{equation*}
			
			\begin{equation*}
				\begin{aligned}
					\implies T(p) &=p'+2p'' \\
								  &=(a_1+4a_2)+(2a_2+12a_3)x+(3a_3+24a_4)x^2+\dots +na_{n}x^{n-}
				\end{aligned}
			\end{equation*}
			
			Sea:
			\begin{equation*}
				q(x)=b_0+b_1x+b_2x^2+\dots +b_{n-1}x^{n-1} \in P_{n-1}
			\end{equation*}
			
		\end{enumerate}
		
		\begin{enumerate}
			
			\item[] \(3.2_{11}\) Sean \(V\) y \(W\) espacios vectoriales de dimensión finita tales que \(\dim V<\dim W\) y \(T:V\to W\) una transformación lineal. Demuestra que \(T\) no es suprayectiva.
			
		\end{enumerate}
		
		Comparando coeficientes:
		\begin{equation*}
			\begin{aligned}
				na_{n} 		 &=b_{n-1} \\
				\implies a_n &= \frac{b_{n-1}}{n} \\
				(n-1)_{an-1}+2n(n-1)_{an} &=b_{n-2} \\
				\implies a_{n-1} &= \frac{b_{n-2-2n(n-1)_{an}}}{n-1} \\
				(n-2)_{an-2}+2(n-1)(n-2)_{an-1} &=b_{n-3} \\
				\implies an-2 &= \frac{b_{n-3}-2(n-1)(n-2)_{an-1}}{n-2} \\
				\vdots \\
				2_{a2}+12_{a3} &=b_1 \\
				a1+4_{a2} &=b_0, a_0 \text{ libre.}
			\end{aligned}
		\end{equation*}
		Entonces:
		\begin{equation*}
			\begin{aligned}
				Im(T) &=P_{n-1}\\
				\dim Im(T) &=n \\
				\text{nulidad }(T) &=1
				\text{, rango }(T) =n \\
				\text{nulidad}(T)+\text{rango}(T)=1+n
				&=n+1 \\
				&=\dim Pn
			\end{aligned}
		\end{equation*}
		
		\[
		\boxed{T\text{, no es inyectiva, ya que el núcleo no es }0.}
		\]
		
		\[
		\boxed{T\text{, si es suprayectiva, ya que su imagen es todo }P_{n-1}.}
		\]
		
		\begin{enumerate}
			
			\item[] \(3.3_{13}\) Sea \(T:\mathbb{R}^2\to\mathbb{R}^2\) tal que \(Ker(T)=Im(T)\)
			
			\begin{equation*}
				\text{a) Demuestra que } T \text{ es una transformación lineal}
			\end{equation*}
			
			P.D. \(T\) es una trasnformación lineal, para cualesquiera polinomios \(f,g\in P (\mathbb{R})\) y cualquier escalar \(\alpha \in \mathbb{R}\) se cumple:
			
			\begin{equation*}
				\begin{aligned}
					T(f+g) = T(f) + T(g) \\
					T(\alpha f) = \alpha, T(f)
				\end{aligned}
			\end{equation*}
			
			Entonces, tomemos \(f,g\in P(\mathbb{R})\). \\
			Sabemos que: 
			\begin{equation*}
				\begin{aligned}
					(f+g)' &=f'+g' \\
					T(f+g) &= (f+g)' \\
						   &= f'+g' \\
					&=T(f)+T(g)
				\end{aligned}
			\end{equation*}
			
			Para $\alpha$ \\
			
			\begin{equation*}
				\text{b) Demuestra que } T \text{ es suprayectiva}
			\end{equation*}
			\begin{equation*}
				\text{c) Muestra con un ejemplo que } T \text{ no es inyectiva}
			\end{equation*}
			
		\end{enumerate}
		
		\begin{enumerate}
			
			\item[] \(3.4_{16}\) Supongamos que \(U\) y \(V\) son espacios vectoriales de dimensión finita y que \(S\in\mathscr{L}(V,W), T\in\mathscr{L}(U,V)\). Demuestra que:
			
			\begin{equation*}
				\dim\text{ }Im(ST) \leq \dim\text{ Ker}S + \dim\text{ Ker}T
			\end{equation*}
			
			\textbf{Hint:} Usar el Teorema de la dimensión y la contención \(Im(ST) \subseteq Im(S)\)
			
		\end{enumerate}
		
		\begin{enumerate}
			
			\item[] \(3.5_{18.a}\) Supongamos que \(\dim V=5\) y que \(S,T \in \mathscr{L}(V)\) son tales que \(ST=0\). Demuestra que:
			
			\begin{equation*}
				\dim\text{ Im}(TS) \leq 2
			\end{equation*}
			
		\end{enumerate}
		
	\end{document}
		